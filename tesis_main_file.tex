\documentclass[12pt,letterpaper]{report}
\usepackage[T1]{fontenc}
\usepackage[utf8]{inputenc}
\usepackage[spanish]{babel}
\decimalpoint
\usepackage{amsmath}
\usepackage{amsfonts}
\usepackage{textgreek}
\usepackage{amssymb}
\usepackage{supertabular}
\usepackage{float}
\usepackage{graphicx}
\usepackage{subfiles}
\usepackage{subcaption}
%\usepackage{kpfonts}
\usepackage[left=2cm,right=2cm,top=2cm,bottom=2cm]{geometry}
\usepackage{multirow}
\usepackage{xcolor}
\usepackage[hidelinks]{hyperref}
\graphicspath{ {figuras/} }
\title{
	{Inestabilidades electrostáticas debidas a haces de partículas.}\\
	{\large UNAM}\\
	{\includegraphics{Escudo-UNAM-transparente.png}}
}
\author{Guillermo Xchell Calva García}
\date{}


\pagenumbering{roman}
\newcommand{\onlyinsubfile}[1]{#1}
\begin{document}
\renewcommand{\onlyinsubfile}[1]{}
\maketitle
%\addcontentsline{toc}{chapter}{Resumen}
%\chapter*{Resumen}
%Abstract goes here
%
%\chapter*{Dedication}
%To mum and dad
%
%\chapter*{Declaration}
%I declare that..
%
%\chapter*{Agradecimientos}
%I want to thank...

\tableofcontents
\listoffigures
\pagenumbering{arabic}
\fontfamily{ptm}\selectfont
\chapter{Interacción entre partículas y ondas}
\subfile{Capitulo1/cap1.tex}
\chapter{Inestabilidad de dos corrientes}
\subfile{Capitulo2/cap2.tex}
%\pagenumbering{arabic}

\appendix
%\numberwithin{equation}{section}
\chapter{Cálculo de raices}
\subfile{Apendice1/ap1.tex}\label{Ap:raices}

\cleardoublepage\phantomsection
\addcontentsline{toc}{chapter}{Bibliografía}
\bibliographystyle{unsrt}
\bibliography{referencias}


\end{document}