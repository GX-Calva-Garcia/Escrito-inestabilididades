\documentclass[../tesis_main_file.tex]{subfiles}
\begin{document}
\onlyinsubfile{\graphicspath{ {../figuras/} }}
\onlyinsubfile{\pagenumbering{arabic}}
\onlyinsubfile{\setcounter{chapter}{1}}
\onlyinsubfile{\chapter{Amortiguamiento de Landau}}
\section{Introducción}
El capítulo anterior expuso como surgen inestabilidades en la onda en la interacción distintas especies tanto en un plasma estacionario como en un haz incidente así como también la transferencia de energía de la onda a las partículas.\\
El proposito de este capítulo es entonces explicar de donde surgen las inestabilidades en estas interacciones asi como también encontrar una relación entre las inestabilidades y la transferencia de energía. \cite{bellan2008fundamentals}.\\
%\section{Amortiguamiento de Landau}
%En el capítulo anterior se estudiaron diferentes casos particulares para la relación de dispersión. Recordando que la forma general está dada por la ecuación \ref{eq_dispersion} que describe un plasma compuesto por distintas especies con velocidades $\overrightarrow{\textbf{u}}_{\sigma 0}$ y frecuencia de plasma $\omega^2_{p \sigma}$.\\
%Ahora bien, haciendo la suposición de que se tienen iones en reposo como fondo, la densidad de cada especie se puede escribir entonces como $n_{\sigma 0} = f_{\sigma}n_0$ donde $n_0$ es la densidad de los iones de fondo y $f_{\sigma}$ es la densidad parcial de la especie $\sigma$. De tal manera que la suma de las $f_{\sigma}$ da la unidad. Por lo que la ecuación \ref{eq_dispersion} se puede reescribir como:
%\begin{equation}
%\label{eq:dispersion_general_1}
%1 =\omega^2_{p0}\sum_{\sigma}\frac{f_{\sigma}}{(\omega - \overrightarrow{\textbf{k}}\cdot \overrightarrow{\textbf{u}}_{\sigma 0})^2}
%\end{equation}
%Donde $\omega^2_{p0}$ es la frecuencia de plasma asociada a la densidad $n_0$. Esta manera alternativa de escribir la relación de dispersión fue simplemente para facilitar la presentación de otra generalización en donde las especies tienen asociadas todas las velocidades y la densidad correspondiente está determinada por una función de distribución $f(u)$, donde el número total de partículas en el intervalo $(u,u+du)$ está dado por $n_0f(u)du$. La densidad $f_{\sigma}$ es entonces sustituida por $f(u)du$ y la suma por una integral de $-\infty$ a $\infty$.
%\begin{equation}
%\label{eq:dispersion_general_2}
%1= \omega^2_{p0}\int^{\infty}_{-\infty}\frac{f(u)}{(\omega-k u)^2}du
%\end{equation}
%Al integrar por partes:
%\begin{equation}
%1=\omega^2_{p0}\left(\frac{f(u)}{(\omega - ku)k}- \int^{\infty}_{-\infty}\frac{df/du}{(\omega -k u)k}du \right)
%\end{equation}
%Si se asume que $f$ se desvanece en el infinito se tiene entonces:
%\begin{equation}
%1=-\frac{\omega^2_{p0}}{k}\int^{\infty}_{-\infty}\frac{df/du}{\omega -k u}du
%\end{equation}
%Esta forma generalizada presenta un problema ya que hay una singularidad para cuando el valor de $v$ se aproxima al valor de la velocidad de fase de la onda ($\omega/k$).\\
%En el siguiente capítulo se expondrá con mayor detalle el problema de la singularidad y como lidiar con ella. 
Como se vio al final de capítulo anterior la generalización de la relación de dispersión para distintas especies conlleva hacer un tratamiento para una función de distribución de velocidades. 
Al tratar con funciones de distribución del plasma, el cual consiste en partículas cargadas con interacciones a larga distancia, es necesaria una ecuación que describa la evolución temporal de estas funciones. Esta ecuación viene a ser la ecuación de Boltzmann sin el término colisional. \cite{bellan2008fundamentals}\\
\begin{equation}
\label{eq:vlasov_def}
\frac{\partial f}{\partial t} + \frac{d\overrightarrow{\textbf{r}}}{dt}\cdot \frac{\partial f}{\partial \overrightarrow{\textbf{r}}} + \frac{d \overrightarrow{\textbf{p}}}{dt}\cdot \frac{\partial f}{\partial \overrightarrow{\textbf{p}}}=0
\end{equation}
Para el caso electrostático de un plasma en tres dimensiones a la ecuación \ref{eq:vlasov_def} se le agregan los términos de la fuerza de Lorentz y el potencial electrostático. Esta forma es comúnmente conocida como la ecuación de Vlasov.
\begin{equation}
\label{eq:vlasov-poisson_3D}
\frac{\partial f_{\sigma}}{\partial t} + \overrightarrow{\textbf{u}}_{\sigma} \cdot \nabla f_{\sigma} -\frac{q_{\sigma}}{m_{\sigma}}\nabla \Phi \cdot \frac{\partial f_{\sigma}}{\partial \overrightarrow{\textbf{u}}}=0
\end{equation}
Sería entonces lógico pensar que para estudiar las ondas desde el punto de vista de Vlasov basta con realizar un proceso de linealización para obtener la relación de dispersión del plasma. Sin embargo resulta que el método de Fourier arroja una singularidad en el integrando, de la misma manera en la que aparece una singularidad al insertar una función de distribución en la relación de dispersión.\\
%%%%%Poner fallo de Fourier%%%%%%%
\section{Tratamiento de Vlasov}
Para ilustar como el análisis de Fourier arroja una singularidad al momento de tratar el problema se tomará el caso de un plasma sin magnetizar en equilibrio estacionario y con comportamiento Maxwelliano. Donde también se asumirá que los iones no se mueven, esto es $m_i \to \infty$, y que la densidad de los iones es igual a la densidad de los electrones en equlibrio $n_e =n_i$. El campo electrostático es entonces cero en el equlibrio debido a que hay neutralidad de carga en el equlibrio. Además como los iones son inmóviles, todo el estudio de la dinámica será de los electrones.Falta entonces plantear al problema de tal manera que este bien definido y tenga relevancia física. Para ello se asume la distribución en equilibrio de velocidades de los electrones como una distribución Maxwelliana.
\begin{equation}
\label{eq:fourier_distribucion_maxwelliana}
f_0(u)=n_0\frac{1}{\pi^{1/2}V_T}e^{-u^2/V_T^2}
\end{equation}
Donde $V_T^2=2k_BT/m$. Y si además se tiene $E=- \partial \Phi/\partial x$ la ecuación linealizada de Vlasov en una dimensión es de la forma:
\begin{equation}
\label{eq:vlasov_1D_linealizada}
\frac{\partial f_1}{\partial t}+u\frac{\partial f_1}{\partial x}-\frac{q}{m}\frac{\partial \Phi _1}{\partial x}\frac{\partial f_0}{\partial u}=0
\end{equation}
Como Vlasov describe la evolución en el espacio fase $u$ es una variable independiente y por lo tanto si se asume que la dependencia espacial tiene la forma de un modo normal $\sim exp(ikx-i\omega t)$ la ecuación \ref{eq:vlasov_1D_linealizada} queda entonces
\begin{equation}
-i(\omega-ku)f_1- ik\Phi _1 \frac{q}{m}\frac{\partial f_0}{\partial u}=0
\end{equation}
Despejando $f_1$
\begin{equation}
f_1 =-\frac{k}{(\omega -ku)}\frac{q}{m}\frac{\partial f_0}{\partial u}\Phi_1
\end{equation}
Se tiene también quela densidad de los electrones perturbados es
\begin{equation}
\label{eq:densidad_electrones_perturbados}
n_1 = \int^{\infty}_{-\infty} f_1 du=-\frac{q}{m}\Phi_1 \int^{\infty}_{-\infty}\frac{k}{(\omega-ku)}\frac{\partial f_0}{\partial u}du
\end{equation}
La ecuación \ref{eq:densidad_electrones_perturbados} proveé una relación entre $n_1$ y $\Phi _1$. Otra relación entre esos dos términos es dada por la ecuación de Poisson, que linealizada en este caso es
\begin{equation}
\frac{\partial ^2 \Phi _1}{\partial x^2}=-\frac{n_1 q}{\epsilon _0}
\end{equation}
Que al realizar una transformación de Fourier en el espacio resulta en
\begin{equation}
k^2\Phi_1 = \frac{n_1 q}{\epsilon_0}
\end{equation}
Que al despejar $\Phi_1$ y sustituyendo en la ecuación \ref{eq:densidad_electrones_perturbados} se obtiene
\begin{equation}
1= - \frac{q^2}{k^2 m \epsilon_0}\int^{\infty}_{-\infty} \frac{k}{(\omega -ku)}\frac{\partial f_0}{\partial u}du
\end{equation}
Que da la relación de dispersión
\begin{equation}
\label{eq:fourier_dispersion_sin_normalizar}
1 + \frac{q^2}{k^2 m \epsilon_0}\int^{\infty}_{-\infty} \frac{k}{(\omega -ku)}\frac{\partial f_0}{\partial u}du=0
\end{equation}
Ahora bien, sustituyendo la ecuación \ref{eq:fourier_distribucion_maxwelliana} en \ref{eq:fourier_dispersion_sin_normalizar} y definiendo las cantidades adimensionales $\xi = u/V_T$ y $\alpha= \omega / k V_T$ resulta en la expresión
\begin{equation}
1+\frac{1}{2k^2}\frac{q^2n_0}{\epsilon_0 k_BT}\frac{1}{\pi ^{1/2}}\int^{\infty}_{-\infty}\frac{1}{(\alpha -\xi)}\frac{\partial}{\partial \xi}e^{-\xi ^2}du=0
\end{equation}
La cual también se puede reescribir como 
\begin{equation}
1 - \frac{1}{2k^2 \lambda _D^2}\frac{1}{\pi ^{1/2}}\int^{\infty}_{-\infty}\frac{1}{(\xi -\alpha)}\frac{\partial}{\partial \xi}e^{-\xi ^2}d\xi=0
\end{equation}
Que vendría siendo una expresión de la forma:
\begin{equation}
1 + \chi =0
\end{equation}
Por lo que se podría expresar la susceptibilidad de los electrones como:
\begin{equation}
\label{eq:susceptibilidad_Vlasov_fourier}
\chi_e = -\frac{1}{2k^2 \lambda _D^2}\frac{1}{\pi ^{1/2}}\int^{\infty}_{-\infty}\frac{1}{(\xi -\alpha)}\frac{\partial}{\partial \xi}e^{-\xi ^2}d\xi
\end{equation}
A diferencia del tratamiento de fluidos en donde se partían  de las ecuaciones de continuidad, de movimiento y de estado, que no son otra cosa que los momentos de la ecuación de Vlasov, aquí se tiene que solo la ecuación de Vlasov está involucrada. Con esto se puede inferir que Vlasov contiene ya toda la información de esas ecuaciones. Sería lógico pensar que este método es un método más simple y directo para encontrar las suscepibilidades que el tratamiento de fluidos. Sin embargo, de este surge una dificultad matemática.\\
Dicha dificultad se presenta debido a que el denominador del integrando de la ecuación \ref{eq:susceptibilidad_Vlasov_fourier} que define a la susceptibilidad se desvanece cuando $\xi=\alpha$, es decir cuando $\omega = kV_T$. Debido a que no hay manera clara de como lidiar con esa singularidad, la integral sobre $\xi$ no puede ser evaluada y el método de Fourier falla.
\section{Tratamiento de Landau}
Para lidiar con esa singularidad se hace uso de la transformada de Laplace, cuyos detalles se pueden revisar en el apéndice \notinsubfile{\ref{Ap:Laplace}}.\\
Con el fin de tener una imagen clara se asume el caso de ondas electrostáticas en un plasma en tres dimensiones, estacionario en el equilibrio, neutral, sin magnetizar, espacialmente uniforme y en donde los iones pueden moverse.\\
Se asume entonces que la distribución de velocidades en equilibrio de cada especie es una función de distribución Maxwelliana en tres dimensiones.
\begin{equation}
\label{eq:dist_max_3D}
f_{\sigma 0}(\overrightarrow{\textbf{u}})=n_{\sigma 0} \left(\frac{m_{\sigma}}{2\pi k_B T_{\sigma}}\right)^{3/2}exp(-m_{\sigma}u^2/2k_BT_{\sigma})
\end{equation}
Donde el campo eléctrico en equilibrio es cero haciendo entonces que el potencial eléctrico tome el valor de una constante, sea por ejemplo ese valor cero. Además se asume que en el tiempo $t=0$ existe una pequeña perturbación en la función de distribución la cual es dependiente del tiempo por lo que la evolución temporal de la distribución queda caracterizada por:
\begin{equation}
f_{\sigma}(\overrightarrow{\textbf{x}},\overrightarrow{\textbf{u}},t)=f_{\sigma 0}(\overrightarrow{\textbf{u}})+f_{\sigma 1}(\overrightarrow{\textbf{x}},\overrightarrow{\textbf{u}},t)
\end{equation}
Por lo que la ecuación de Vlasov linealizada queda:
\begin{equation}
\frac{\partial f_{\sigma 1}}{\partial t}+ \overrightarrow{\textbf{u}} \cdot \nabla f_{\sigma 1} -\frac{q_{\sigma}}{m_{\sigma}}\nabla \Phi _1 \cdot \frac{f_{\sigma 0}}{\partial  \overrightarrow{\textbf{u}}}=0
\end{equation}
Si se asume que la dependencia espacial de las perturbaciones es de la forma $\sim exp(i\overrightarrow{\textbf{k}}\cdot \overrightarrow{\textbf{x}})$, que viene a ser lo mismo que aplicar una transformada de Fourier en el espacio, se tiene:
\begin{equation}
\frac{\partial f_{\sigma 1}}{\partial t}+ i\overrightarrow{\textbf{k}}\cdot \overrightarrow{\textbf{u}}f_{\sigma 1}-\frac{i q_{\sigma}}{m_{\sigma}}\Phi _1 \overrightarrow{\textbf{k}}\cdot \frac{\partial f_{\sigma 0}}{\partial \overrightarrow{\textbf{u}}}=0
\end{equation}
Es aquí donde se emplea la transformada de Laplace para asi incorporar el valor inicial.
\begin{equation}
(p + i\overrightarrow{\textbf{k}}\cdot \overrightarrow{\textbf{u}})f_{\sigma 1}(\overrightarrow{\textbf{u}},p)- f_{\sigma _1}(\overrightarrow{\textbf{u}},0)-\frac{i q_{\sigma}}{m_{\sigma}}\Phi _1(p)\overrightarrow{\textbf{k}}\cdot \frac{f_{\sigma 0}}{\partial \overrightarrow{\textbf{u}}}
\end{equation}
Si se despeja $f_{\sigma 1}(\overrightarrow{\textbf{u}},p)$ se tiene entonces:
\begin{equation}
\label{eq:f1_despeje_laplace}
f_{\sigma 1}(\overrightarrow{\textbf{u}},p)=\frac{1}{(p + i\overrightarrow{\textbf{k}}\cdot \overrightarrow{\textbf{u}})}\left[\frac{i q_{\sigma}}{m_{\sigma}}\Phi _1(p)\overrightarrow{\textbf{k}}\cdot \frac{f_{\sigma 0}}{\partial \overrightarrow{\textbf{u}}} + f_{\sigma _1}(\overrightarrow{\textbf{u}},0)\right]
\end{equation}
En este marco se puede escribir la ecuación de Poisson como:
\begin{equation}
\nabla ^2 \Phi _1 =- \frac{1}{\epsilon_0}\sum_{\sigma} q_{\sigma}n_{\sigma 1}= - \frac{1}{\epsilon_0}\sum q_{\sigma} \int f_{\sigma 1}(\overrightarrow{\textbf{x}},\overrightarrow{\textbf{u}},t)d^3u
\end{equation}
Sustituyendo $\nabla \rightarrow i \overrightarrow{\textbf{k}}$ y aplicando la transformda de Laplace se llega a:
\begin{equation}
\label{eq:poisson_sustitucion_Laplace}
k^2 \Phi _1(p)=\frac{1}{\epsilon_0} \sum_{\sigma} q_{\sigma}\int f_{\sigma 1}(\overrightarrow{\textbf{u}},p)d^3u
\end{equation}
Entonces, al sustituir la ecuación \ref{eq:f1_despeje_laplace} en \ref{eq:poisson_sustitucion_Laplace} se tiene
\begin{equation}
k^2 \Phi _1(p)=\frac{1}{\epsilon_0}\sum_{\sigma}q_{\sigma}\int \left[\frac{\frac{i q_{\sigma}}{m_{\sigma}}\Phi _1(p)\overrightarrow{\textbf{k}}\cdot \frac{f_{\sigma 0}}{\partial \overrightarrow{\textbf{u}}} + f_{\sigma _1}(\overrightarrow{\textbf{u}},0)}{(p + i\overrightarrow{\textbf{k}}\cdot \overrightarrow{\textbf{u}})}\right]d^3u
\end{equation}
De donde se factoriza $\Phi_1$ 
\begin{equation}
\Phi _1 (p)\left[1-\frac{1}{\epsilon_0 k^2}\sum_{\sigma}\frac{q_{\sigma}^2}{m_{\sigma}}\int  \frac{i\overrightarrow{\textbf{k}}\cdot \frac{\partial f_{\sigma 0}}{\partial \overrightarrow{\textbf{u}}}}{(p + i\overrightarrow{\textbf{k}}\cdot \overrightarrow{\textbf{u}})}d^3u \right]=\frac{1}{\epsilon_0 k^2}\sum_{\sigma}q_{\sigma}\int \frac{f_{\sigma 1}(\overrightarrow{\textbf{u}})}{(p + i\overrightarrow{\textbf{k}}\cdot \overrightarrow{\textbf{u}})}d^3u
\end{equation}
Entonces se tiene una expresión para $\Phi_1(p)$ de la forma
\begin{equation}
\label{eq:phi_num_y_denominador}
\Phi_1(p)=\frac{N(p)}{D(p)}
\end{equation}
Donde:
\begin{equation}
\label{eq:funcion_numerador}
N(p)=\frac{1}{\epsilon_0 k^2}\sum_{\sigma}q_{\sigma}\int \frac{f_{\sigma 1}(\overrightarrow{\textbf{u}})}{(p + i\overrightarrow{\textbf{k}}\cdot \overrightarrow{\textbf{u}})}d^3u
\end{equation}
y
\begin{equation}
\label{eq:funcion_denominador}
D(p)=1-\frac{1}{\epsilon_0 k^2}\sum_{\sigma}\frac{q_{\sigma}^2}{m_{\sigma}}\int  \frac{i\overrightarrow{\textbf{k}}\cdot \frac{\partial f_{\sigma 0}}{\partial \overrightarrow{\textbf{u}}}}{(p + i\overrightarrow{\textbf{k}}\cdot \overrightarrow{\textbf{u}})}d^3u
\end{equation}
De la construcción de la transformada inversa de Laplace, apéndice \notinsubfile{\ref{Ap:Laplace}}, la expresión para $\Phi _1(t)$ es:
\begin{equation}
\label{eq:phi_inversa_laplace}
\Phi_1(t) =\frac{1}{2\pi i}\int ^{\beta +i \infty}_{\beta - i\infty} \frac{N(p)}{D(p)}e^{pt}dt
\end{equation}
Con $\beta$ más grande que el término exponencial contenido en $N(p)/D(p)$ que crece con mayor rapidez.\\
Se tiene entonces en la ecuación \ref{eq:phi_inversa_laplace} una solución exacta para $\Phi_1$, sin embargo debido a la complejidad de $N(p)$ y $D(p)$ no es una expresión que sea fácil de evaluar. Por otro lado, si se trata el límite asintótico para tiempos largos se vuelve un problema más manejable y arroja luz sobre la física que hay detrás.\\
Pero incluso antes de empezar el estudio de ese caso son necesarias algunas herramientas de la teoría de variable compleja para lidiar con los posibles polos de la ecuación \ref{eq:phi_inversa_laplace} en particular la continuación analítica de una función. Los detalles referentes a los polos, residuos y a la continuación analítica de una función se pueden revisar en el apéndice \notinsubfile{\ref{Ap:temas_Var_compleja}}.\\
Debido a la forma de la ecuación \ref{eq:phi_inversa_laplace} es razonable esperar que presente algunos polos que dificulten la evaluación de la integral. Además, la ecuación \ref{eq:phi_inversa_laplace} presenta otro problema: tanto $N(p)$ como $D(p)$ no están definidos para la región $Re(p) < \gamma$ debido a que se llegó a estos términos por medio de transformadas de Laplace las cuales solo están definidas para la región $Re(p) > \gamma$, donde $\gamma$ es el factor exponencial que crece con mayor rapidez en la función a la que se le está aplicando la transformada, véase el apéndice \notinsubfile{\ref{Ap:Laplace}}. Por lo que es necesaría la construcción de continuaciones analíticas para $N(p)$ y $D(p)$.\\
La manera más simple de hacer esto es simplemente extendiendo la definición a todo el plano complejo de $p$ en las expresiones para $N(p)$ y$D(p)$ con la condición de que las nuevas funciones permanezcan analíticas bajo está nueva extensión.\\
Considérese primero la continuación analítica de $N(p)$
\begin{equation}
\label{eq:cont_analitica_N(p)}
N(p) = \frac{1}{k^2 \epsilon_0}\sum_{\sigma}q_{\sigma}\int ^{\infty}_{-\infty}\frac{F_{\sigma 1}(u_{\parallel},0)}{(p +iku_{\parallel})}du_{\parallel}=\frac{1}{ik^3 \epsilon_0}\sum_{\sigma}q_{\sigma}\int ^{\infty}_{-\infty}\frac{F_{\sigma 1}(u_{\parallel},0)}{(u_{\parallel} -ip/k)}du_{\parallel}
\end{equation}
Donde $\parallel$ indica que es en la dirección $\overrightarrow{\textbf{k}}$, mientras que a la componente paralela del valor inicial de la función de distribución perturbada se le define como
\begin{equation}
F_{\sigma 1}(u_{\parallel},0)=\int f_{\sigma 1}(\overrightarrow{\textbf{u}},0)d^2\overrightarrow{\textbf{u}}_{\perp}
\end{equation}
El integrando de la ecuación \ref{eq:cont_analitica_N(p)} tiene entonces un polo en $u_{\parallel}=ip/k$. Se toma además a $k>0$ y se recuerda que $N(p)$ tiene la restricción de que $\operatorname{Re}(p)$ debe ser siempre mayor que $\gamma$ para que el polo $u_{\parallel}=ip/k$ se encuentre en la mitad superior del pano complejo de $u_{\parallel}$. Al realizar la continuación analítica se puede permitir a $\operatorname{Re}(p)$ ser menor que $\gamma$ en incluso negativa lo cual permite al polo $u_{\parallel}=ip/k$ moverse en dirección de la parte inferior del plano complejo de $u_{\parallel}$. Surgue sin embargo el problema de como lidiar con el polo cuando este pasa por el cero pues pasaría a estar por debajo del controno de integración y $N(p)$ dejaría de ser analítica. Landau propusó que el contorno se deforme a medida que $\operatorname{Re}(p)$ se vaya haciendo negativa de tal manera que el contorno siempre se encuentre por debajo del polo.\cite{bellan2008fundamentals}\\
La continuación analítica de $D(p)$ sigue el mismo procedimiento y es también construida a partir de la deformación del contorno de integración de tal manera que siempre se encuentre por debajo del polo.
\begin{equation}
\label{eq:cont_analitica_D(p)}
D(p)=1-\frac{1}{k^2}\sum _{\sigma}\frac{q_{\sigma}^2}{\epsilon _0m_{\sigma}}\int ^{\infty}_{-\infty}\frac{\partial F_{\sigma 0}/\partial u_{\parallel}}{(u_{\parallel} -ip/k)}du_{\parallel}
\end{equation}
Una vez definidas las continuaciones analíticas de $N(p)$ y $D(p)$ hacia la izquierda del plano complejo de $p$ el cálculo de la integral en la ecuación \ref{eq:phi_inversa_laplace} pasa a ser una tarea más sencilla pues permite usar el método del residuo.\\
En el caso general donde $N(p)/D(p)$ tenga varios polos en la parte izquierda del plano complejo de $p$ el controno se puede deformar de tal manera que la parte vertical se pueda recorrer hacia la izquierda deformandose alrededor de un un polo $p_j$ cuando pase por este. A medida que el contorno se va moviendo hacia la izquierda es importante recordar que cuando $\operatorname{Re}(p)\rightarrow - \infty$ el numerador tiende a cero mientras que el deniminador tiende a uno. Como $exp(pt)\rightarrow 0$ para $\operatorname{Re}(p)\rightarrow - \infty$ y tiempos positivos la línea vertical recorrida hacia la izquierda no contribuye a la integral por lo que la evaluación de la ecuación \ref{eq:phi_inversa_laplace} consiste en la summa de los residuos de todos sus polos.
\begin{equation}
\label{eq:Phi_suma_residuos}
\Phi_1 (t)=\sum_j \lim _{p\to p_j}\left[(p-p_j)\frac{N(p)}{D(p)}e^{pt}\right]
\end{equation}
Queda entonces la incognita del origen de estos polos, para la cual se tienen dos posibiliades: $N(p)$ tiene un polo explícito esto es que tenga un término $\sim 1 /(p-p_j)$ o bién $D(p)$ contiene un factor $\sim (p-p_j)$ es decir $p_j$ es un cero de $D(p)$.\\
Se podría asumir que el polo de $N(p)$ es el que esta contenido en el integrando pero este polo se utiliza como residuo al momento de integrar por lo que no contribuye. Por otro lado, la integral sobre $v_{\parallel}$ de $F_{\sigma 1}(u_{\parallel},0)$ es finita y por lo tanto tampoco contribuye. Lo anterior deja claro que los polos se encuentran en los ceros de $D(p)$.\\
Con el fin de deja más clara la física detrás de esto se puede realizar una simplificación más. Esto es tomar el problema solamente para el comportamiento asintótico de tiempos largos donde  el polo $p_j$ ubicado más hacia la derecha domina sobre las contribuciones de los otros polos, por lo cuál lo único que se requiere es encontrar la raíz $p_j$ que tenga la parte real más grande. Y como se había mencionado antes tal raíz será un cero de la ecuación \ref{eq:cont_analitica_D(p)}.\\
Si se observa además que las cantidades que se están sumando en la ecuación \ref{eq:cont_analitica_D(p)} son en escencia las perturbaciones de los iones y electrones asociadas con la oscilación se puede escribir entonces
\begin{equation}
\label{eq:D(p)_como_suma_de_susceptibilidades}
D(p)=1 + \sum_{\sigma}\chi_{\sigma}=0
\end{equation}
Esto también resulta evidente si se compara la ecuación \ref{eq:cont_analitica_D(p)} con la ecuación \ref{eq:susceptibilidad_Vlasov_fourier} que se había obtenido antes para expresar susceptibilidades. Con esto, se hace uso de que la función de distribución en equilibrio es Maxwelliana y entonces se expresan a las susceptibilidades como:
\begin{multline}
\label{eq:susceptibilidad_Landau}
\chi_{\sigma} = -\frac{1}{2k^2 \lambda ^2_{D\sigma}}\frac{1}{\pi^{1/2}}\int ^{\infty}_{-\infty}\frac{1}{(\xi - ip/kV_{T\sigma})}\frac{\partial}{\partial \xi}exp(-\xi ^2)d\xi\\
= \frac{1}{k^2 \lambda ^2_{D\sigma}}\left[ \frac{1}{\pi ^{1/2}}\int ^{\infty}_{-\infty} \frac{(\xi - ip/kV_{T\sigma}+ ip/kV_{T\sigma})}{(\xi - ip/kV_{T\sigma})}exp(-\xi ^2)d\xi\right]\\
=\frac{1}{k^2 \lambda ^2_{D\sigma}}\left[1 + \frac{1}{\pi ^{1/2}}\alpha\int^{\infty}_{-\infty}\frac{exp(-\xi ^2)}{\xi -\alpha}d\xi\right]\\
= \frac{1}{k^2 \lambda ^2_{D\sigma}}\left[1+\alpha Z(\alpha)\right]
\end{multline}
Donde $\alpha = ip/kV_{T\sigma}$ y el últmo renglón introduce un término referente a una función de dispersión del plasma $Z(\alpha)$ definida como:
\begin{equation}
\label{eq:Z_dispersion_plasma}
Z(\alpha) =\frac{1}{\pi ^{1/2}}\int^{\infty}_{-\infty}\frac{exp(-\xi ^2)}{(\xi- \alpha)}d\xi
\end{equation}
\section{Evaluación de la función de dispersión del plasma.}
Para la evaluación de la ecuación \ref{eq:Z_dispersion_plasma} se pueden considerar dos casos límites. El primero es cuando se tiene $|\alpha|\gg 1$, que corresponde al límite adiabático en el análisis de fluidos donde $\omega / k \gg V_{T\sigma}$, y cuando $|\alpha|\ll 1$ que su vez corresponde al límite isotérmico en el análisis de fluidos donde $\omega / k \ll V_{T\sigma}$\\
La correspondencia de estos límites para $|\alpha |$ con los límites adiabático e isotérmico obtenidos del análisis de fluidos sugiere entonces que para esos dos casos se puede hacer uso de ciertos resultados de este al momento de evaluar $Z(\alpha )$. En particular se tiene que el análisis de fluidos solo habían ondas no amortiguadas por lo que las perturbaciones que aparecen como resultado del análisis de Vlasov deben de ser lo suficientemente debiles para que haya una correspondencia en estos límites. Lo anterior es equivalente a decir que la solución de Vlasov correspondiente al modo del análisis de fluidos solo puede tener un polo que de encontrarse debajo del eje real solo sea a una distancia lo suficientemente pequeña. Por lo tanto solo es necesario hacer una continuación analítica de $N(p)/D(p)$ ligeramente hacia la parte negativa del eje imaginario de $p$ y por lo tanto tener un contorno de integración que se deforme solo ligeramente hacia la parte negativa del eje imaginario, por ejemplo un semicírculo de radio $\delta$ que pase por debajo del polo. Esto a su vez permite dividir el contorno de integración para la ecuación \ref{eq:Z_dispersion_plasma} en tres partes:
\begin{itemize}
\item Una línea con dirección hacia la derecha $\xi \in (-\infty , \alpha -\delta ]$ pero ubicada a la izquierda del polo.
\item Un semicírculo de radio $\delta$ debajo del polo en sentido contrario a las manecillas del reloj y que solamente encierre parcialmente al polo.
\item Una línea que parte de $\alpha +\delta$, es decir a la derecha del polo, y continúa hacia $\xi < \infty$.
\end{itemize}
En el límite $\delta \to 0$ la suma de los dos segmentos que recorren una línea definen lo que se conoce como el valor principal de la integral. Por otro lado, la parte correspondiente al semicírculo resulta ser medio residuo por lo que su contribución es $i\pi$ veces este, en lugar de $2i\pi$. Con esto se tiene entonces que la función de dispersión del plasma para un polo que se encuentra ligeramente por debajo del eje real es:
\begin{equation}
\label{eq:Z_ValorP_medioRes}
Z(\alpha) = \frac{1}{\pi ^{1/2}}\left[P\int ^{\infty}_{-\infty} \frac{exp(-\xi ^2)}{(\xi -\alpha)}d\xi\right]+ i\pi^{1/2}exp(-\alpha ^2)
\end{equation}
Donde $P$ denota el valor principal de la integral y el sgundo término del lado derecho expresa la contribución del semicírculo. La ecuación \ref{eq:Z_ValorP_medioRes} da una manera de lidiar con las integrales mal definidas del tipo como la que aparece en la ecuación \ref{eq:fourier_dispersion_sin_normalizar}.\\
Regresando entonces al problema de los dos casos límites se empezará primero revisando el caso $\alpha \gg 1$ que corresponde al caso de ondas de electrones.\\
Primero se empieza por notar que el factor $exp(-\xi^2)$ solo contribuye de manera significativa a la integral cuando $\xi$ es del orden de la unidad o más pequeña. En la parte de la integral donde este término es finito, la cual es la región de interes, se tiene $|\alpha | \gg \xi$. 
Es entonces en esta región donde los otros factores del integrando pueden ser expandidos de la siguiente manera
\begin{equation}
\frac{1}{(\xi - \alpha)}= -\frac{1}{\alpha}\left(1-\frac{\xi}{\alpha}\right)^{-1}=-\frac{1}{\alpha}\left[1+\frac{\xi}{\alpha}+\left(\frac{\xi}{\alpha}\right)^2+\left(\frac{\xi}{\alpha}\right)^3+\left(\frac{\xi}{\alpha}\right)^4+\cdots\right]
\end{equation}
Sustituyendo en la integral de la ecuación \ref{eq:Z_ValorP_medioRes} y haciendo uso del hecho de que se trata de una integral par, se desprecian los términos impares pues estos no contribuyen y se llega a la siguiente expresión:
\begin{equation}
P\int ^{\infty}_{-\infty} \frac{exp(-\xi ^2)}{(\xi -\alpha)}d\xi = - \frac{1}{\alpha}\frac{1}{\pi ^{1/2}}\int^{\infty}_{-\infty}exp(-\xi^2)d\xi \left[1 + \left(\frac{\xi}{\alpha}\right)^2 + \left(\frac{\xi}{\alpha}\right)^4 + \cdots \right]
\end{equation}
Los términos resultantes sugieren una integral de forma Gaussiana, o bien la suma de varias integrales de forma Gaussiana, de lo cual se puede sacar provecho si se toman las derivadas sucesivas con respecto de $a$ de la expresión
\begin{equation*}
\frac{1}{\pi ^{1/2}}\int exp(-a\xi ^2)d\xi =\frac{1}{a^{1/2}}
\end{equation*}
De lo cual haciendo $a=1$ se tiene
\begin{align*}
\frac{1}{\pi ^{1/2}}\int \xi ^2 exp(-a\xi ^2)d\xi =\frac{1}{2}& &\frac{1}{\pi ^{1/2}}\int \xi ^4 exp(-a\xi ^2)d\xi =\frac{3}{4}
\end{align*}
El valor principal toma entonces la forma 
\begin{equation}
\label{eq:ValorP_alpha_gg_1}
P\int ^{\infty}_{-\infty} \frac{exp(-\xi ^2)}{(\xi -\alpha)}d\xi = -\frac{1}{\alpha}\left[1 + \frac{1}{2\alpha^2}+\frac{3}{4\alpha^4}+\cdots\right]
\end{equation}
Por lo que la función de dispersión del plasma en el caso asintótico $|\alpha|\gg 1$ tiene la forma
\begin{equation}
\label{eq:Z_aprox_alpha_gg_1}
Z(\alpha)= -\frac{1}{\alpha}\left[1 + \frac{1}{2\alpha^2}+\frac{3}{4\alpha^4}+\cdots\right]+ i\pi ^{1/2}exp(-\alpha)
\end{equation}
Para evaluar la integral en el límite de $|\alpha |\ll 1$ se introduce una variable $\eta = \xi -\alpha$ donde $d\eta = d\xi$ por lo que el valor principal de la ecuación \ref{eq:Z_ValorP_medioRes} se puede reescribir y expandir de la siguiente manera:
\begin{equation}
\begin{split}
\frac{1}{\pi^{1/2}}\int^{\infty}_{-\infty}\frac{exp(-\eta ^2 - 2\alpha \eta - \alpha^2 )}{\eta}d\eta &=\\
&=\frac{e^{-\alpha^2}}{\pi^{1/2}}\int^{\infty}_{-\infty}\frac{e^{-\eta^2}}{\eta}d\eta \left[1-2\alpha \eta + \frac{(-2\alpha)^2}{2!}+\frac{(-2\alpha)^3}{3!}+\cdots\right]\\
&=-2\alpha\frac{e^{-\alpha^2}}{\pi^{1/2}}\int^{\infty}_{-\infty}e^{-\eta^2}d\eta\left[1+\frac{2\eta^2\alpha^2}{3}+\cdots\right]\\
&=-\alpha \left(1-\alpha^2+\cdots\right)\left(1+\frac{\alpha^2}{3}+\cdots\right)\\
&=-2\alpha \left(1-\frac{2\alpha^2}{3} + \cdots \right)
\end{split}
\end{equation}
Donde los términos impares del segundo renglón se integran a cero en el tercer rengón debido a la simetría de la integral.
Teniendo entonces que la función de dispersión del plasma para este límite es
\begin{equation}
\label{eq:Z_aprox_alpha_ll_1}
Z(\alpha)=-2\alpha \left(1-\frac{2\alpha^2}{3}+\cdots \right)+i\pi^{1/2}exp(-\alpha^2)
\end{equation}
Los límites anteriores corresponden al caso físico de ondas de electrones y ondas acústicas de iones los cuales a su vez hablan de diferentes susceptibilidades y por consecuencia de diferentes comportamientos para los amortiguamientos que surgen de estudiar las oscilaciones.
\section{Ondas de electrones}
Esto corresponde al caso $|\alpha| \gg 1$ por lo cual se utiliza la ecuación \ref{eq:Z_aprox_alpha_gg_1} para la evaluación de \ref{eq:susceptibilidad_Landau} y donde además se define una suerte de frecuencia $\omega = ip$ para de esa manera tener $\alpha = \omega /kV_{T\sigma}$ y  $\alpha_i = \omega_i /kV_{T\sigma}$ por lo que entonces la susceptibilidad puede expresarse como:
\begin{equation}
\label{eq:susceptibilidad_landau_ondas_electrones}
\begin{split}
\chi _{\sigma} &= \frac{1}{k^2 \lambda^2_{D\sigma}}\left\lbrace 1+\alpha \left[-\frac{1}{\alpha}\left(1+\frac{1}{2\alpha^2}+\frac{3}{4\alpha^4}+\cdots\right)+i\pi^{1/2}exp(-\alpha^2)\right]\right\rbrace\\
&=\frac{1}{k^2 \lambda^2_{D\sigma}}\left\lbrace - \left(\frac{1}{2\alpha^2}+\frac{3}{4\alpha^4}+\cdots\right) + i\alpha\pi^{1/2}exp(-\alpha^2)\right\rbrace\\
&=-\frac{\omega^2_{p\sigma}}{\omega^2}\left(1+3\frac{k^2}{\omega^2}\frac{k_BT_{\sigma}}{m_{\sigma}}+\cdots \right)+i\frac{\omega}{kV_{T\sigma}}\frac{\pi^{1/2}}{k^2\lambda^2_{D\sigma}}exp(-\omega^2/k^2V^2_{T\sigma})
\end{split}
\end{equation}
Por lo que si se se tiene que la raíz es tal que $|\alpha |\gg 1$ la ecuación \ref{eq:D(p)_como_suma_de_susceptibilidades} para los polos pasa a ser
\begin{equation}
\label{eq:suma_susc_ondas_electrones}
1-\sum_{\sigma}\frac{\omega^2_{p\sigma}}{\omega^2}\left(1+3\frac{k^2}{\omega^2}\frac{k_BT_{\sigma}}{m_{\sigma}}+\cdots \right)+i\frac{\omega}{kV_{T\sigma}}\frac{\pi^{1/2}}{k^2\lambda^2_{D\sigma}}exp(-\omega^2/k^2V^2_{T\sigma}) =0
\end{equation}
Que resulta ser similar a la expresión encontrada para la relación de dispersón en el tratamiento de fluidos con la excepción de que en este caso se tienen términos adicionales imaginarios.
Se tiene además que la ecuación \ref{eq:suma_susc_ondas_electrones} no es una relación de dispersión sino la ecuación para las raíces de $D(p)$. Las cuales a su vez determinan los polos en $N(p)/D(p)$ que producen las oscilaciones con menor amortiguamiento, reultantes de alguna perturbación inicial en la función de distribución.\\
Como $\omega^2_{pe}/\omega^2_{pi}=m_i/m_e$ y además en genral se tiene $V_{Ti} \ll V_{Te}$ tanto la parte real como la imaginaria referente a los términos de los iones son mucho más pequñas que los términos de electrones. Por lo tanto, despreciando los términos de los iones
\begin{equation}
\label{eq:relacion_disp_ondas_electrones}
1- \frac{\omega^2_{pe}}{\omega^2}\left(1+3\frac{k^2}{\omega^2}\frac{k_BT_{e}}{m_{e}}+\cdots \right)+i\frac{\omega}{kV_{Te}}\frac{\pi^{1/2}}{k^2\lambda^2_{De}}exp(-\omega^2/k^2V^2_{Te}) =0
\end{equation}
Recordando que $\omega$ es una cantidad compleja se separa entonces en su parte real e imaginaria $\omega =\omega_r +\omega_i$ y de esta manera se prosigue a encontrar la raíz compleja $\omega$ de \ref{eq:relacion_disp_ondas_electrones}.
Para ello se reescribe la ecuación como
\begin{equation}
\label{eq:D_Dr_mas_Di}
D(\omega_r + i\omega_i) = D_r(\omega_r + i\omega_i)+iD_i(\omega_r + i\omega_i)=0
\end{equation}
Con $D_r$ la parte de $D$ que no contiene ningún término imaginario explícito mientras que $D_i$ es la parte que contiene de manera explícita a $i$. De esta manera se tiene:
\begin{align}
\label{eq:Dr_y_Di_ondas_electrones}
D_r =1-\frac{\omega^2_{pe}}{\omega^2}\left(1+\frac{3k^2}{\omega^2}\frac{k_BT_e}{m_e}+\cdots\right)&&D_i=\frac{\omega}{kV_{Te}}\frac{\pi^{1/2}}{k^2\lambda^2_{De}}exp(-\omega^2/k^2V^2_{Te})
\end{align}
Además al haber asumido amortiguaciones debiles, es decir $\omega_1 \ll \omega_r$ la ecuación \ref{eq:D_Dr_mas_Di} puede expandirse por Taylor alrededor de $\omega_i$
\begin{equation}
\label{eq:expansion_Dr_ondas_electrones}
D_r(\omega_r)+i\omega_i \left(\frac{dD_r}{d\omega}\right)_{\omega=\omega_r} + i\left[D_i(\omega_r)+i\omega_i\left(\frac{dD_i}{d\omega}\right)_{\omega=\omega_r}\right]
\end{equation}
Como $\omega_i \ll \omega_r$ la parte real de la ecuación \ref{eq:expansion_Dr_ondas_electrones} es aproximadamente cero y por lo tanto se desprecia.
\begin{equation}
\label{eq:Dr_ondas_electrones}
D_r(\omega_r)\simeq 0
\end{equation}
Por otro lado si se despeja $\omega_i$ de la parte imaginaria se obtiene
\begin{equation}
\label{eq:despeje_parte_im_ondas_electrones}
\omega_i=-\frac{D_i(\omega_r)}{\frac{dD_r}{d\omega}}
\end{equation}
Por lo que las ecuaciones \ref{eq:Dr_ondas_electrones} y \ref{eq:Dr_y_Di_ondas_electrones} dan la parte real de la frecuencia como
\begin{equation}
\label{eq:wr_ondas_electrones}
\omega^2_r=\omega^2_{pe}\left(1+3\frac{k^2}{\omega^2_r}\frac{k_BT_e}{m_e}\right)\simeq\omega^2_{pe}\left(1+3k^2\lambda^2_{De}\right)
\end{equation}
Mientras que las ecuaciones \ref{eq:despeje_parte_im_ondas_electrones} y \ref{eq:Dr_y_Di_ondas_electrones} dan la parte imaginaria
\begin{equation}
\label{eq:Amortiguamiento_Landau_ondas_electrones}
\begin{split}
\omega_i &=-\sqrt{\frac{\pi}{8}}\frac{\omega_{pe}}{k^3\lambda^3_{De}}exp(-\omega^2/k^2V^2_{Te})\\
&=-\sqrt{\frac{\pi}{8}}\frac{\omega_{pe}}{k^3\lambda^3_{De}}exp\left[-(1+3k^2\lambda^2_{De})/2k^2\lambda^2_{De}\right]
\end{split}
\end{equation}
Es justamente a esa parte imaginaria que se le conoce como el amortiguamiento de Landau. Debido a que las oscilaciones menos amortiguadas evolucionan de la forma $exp(pt)=exp(-i\omega_rt+\omega_it)$ y en la ecuación \ref{eq:Amortiguamiento_Landau_ondas_electrones} se tiene una $\omega_i$ negativa, queda claro entonces que se trata de un amortiguamiento.\\
Por último cabe señalar que si bien se asumió $\omega_i \ll \omega_r$ no quiere decir que $\omega_i$ sea despreciable debido a que el factor $2\pi$ afecta de manera diferente las partes imaginarias y reales de la fase de la onda.
\section{Ondas acústicas de iones}
Las ondas acústicas de los iones son un resultado del análisis de dos fluidos en la región donde $V_{Ti} \ll \omega /k \ll V_{Te}$. En donde los electrones exhiben un comportamiento isotérmico mientras que los iones se comportan de manera adibática.
Esta diferencia de comportamiento siguiere que $D(p)$ también tiene raíces en esa región donde $|\alpha _e|\ll 1$ y $|\alpha _i|\gg 1$.\\
Se tiene entonces que de las ecuaciones \ref{eq:susceptibilidad_Landau} y \ref{eq:Z_aprox_alpha_ll_1} la suceptibilidad en el caso $|\alpha | \ll 1$ es
\begin{equation}
\label{eq:susceptibilidad_Landau_ondas_iones}
\begin{split}
\chi _{\sigma} &= \frac{1}{k^2\lambda^2_{D\sigma}}\left\lbrace 1-2\alpha^2 \left(1-\frac{2\alpha^2}{3}+\cdots \right) + i\alpha \pi^{1/2}exp(-\alpha^2)\right\rbrace \\
&\simeq \frac{1}{k^2\lambda^2_{D\sigma}}+i\frac{\alpha}{k^2\lambda^2_{D\sigma}}\pi^{1/2}exp(-\alpha^2)
\end{split}
\end{equation}
Si además se hace uso de las ecuaciones \ref{eq:susceptibilidad_Landau_ondas_iones} para la suceptibiliad de los electrones y \ref{eq:susceptibilidad_landau_ondas_electrones} para la susceptibiliad de los iones se obtiene
\begin{equation}
\begin{split}
D(\omega) &= 1+\frac{1}{k^2 \lambda^2_{De}}+i\frac{\omega}{kV^2_{Te}}\frac{\pi^{1/2}}{k^2\lambda^2_{De}}exp(-\omega^2/k^2V^2_{Te})\\
&-\frac{\omega^2_{pi}}{\omega^2}\left(1+3\frac{k^2}{\omega^2}\frac{k_BT_i}{m_i}+\cdots \right)+i\frac{\omega}{kV_{Ti}}\frac{\pi^{1/2}}{k^2\lambda^2_{Di}}exp(-\omega^2/k^2V^2_{Ti})
\end{split}
\end{equation}
Si se hace una expansión de Taylor de la misma manera que en las ecuaciones \ref{eq:Dr_ondas_electrones} y \ref{eq:despeje_parte_im_ondas_electrones} se llega a que $\omega_r$ es la raíz de 
\begin{equation}
D_r(\omega_r)=1+\frac{1}{k^2\lambda^2_{De}}-\frac{\omega^2_{pi}}{\omega^2_r}\left(1+3\frac{k^2}{\omega^2}\frac{k_BT_i}{m_i}\right)=0
\end{equation}
y por lo tanto
\begin{equation}
\omega^2_r =\frac{k^2 C^2_s}{1+k^2\lambda^2_{De}}\left(1+3\frac{k^2}{\omega^2}\frac{k_BT_i}{m_i}\right)\simeq\frac{k^2C^2_s}{1+k^2\lambda^2_{De}}+3k^2\frac{k_BT_i}{m_i}
\end{equation}
Donde $C^2_s=\omega^2_{pi}\lambda^2_{De}=k_BT_e/m_i$ es la velocidad acústica de los iones. Por otro lado la perte imaginaria resulta ser
\begin{equation}
\label{eq:Amortiguamiento_Landau_ondas_iones}
\begin{split}
\omega_i &\simeq -\frac{D_i(\omega_r)}{\frac{dD_r}{d\omega}}\\
&= -\frac{\omega \pi^{1/2}}{k^3}\left[\frac{\frac{1}{\lambda^2_{De}}exp(-\omega^2/k^2V^2_{Te})+\frac{1}{\lambda^2_{Di}V_{Ti}}exp(-\omega^2 /k^2V^2_{Ti}}{2\omega^2_{pi}/\omega^3}\right]\\
&=-\frac{\omega^4}{k^3C^3_s}\sqrt{\frac{\pi}{8}}\left[\sqrt{\frac{m_e}{m_i}}+\left(\frac{T_e}{T_i}\right)^{3/2}exp(-\omega^2/k^2V^2_{Ti})\right]\\
&=-\frac{|\omega_r|}{(1+k^2\lambda^2_{De})^{3/2}}\sqrt{\frac{\pi}{8}}\left[\sqrt{\frac{m_e}{m_i}}+\left(\frac{T_e}{T_i}\right)^{3/2}exp\left(-\frac{T_e/2T_i}{1+2k^2\lambda^2_{De}}-\frac{3}{2}\right)\right]
\end{split}
\end{equation}
Resulta que el amortiguamiento dominante de Landau proviene de los iones, esto es debido a que el amortiguamiento de Landau asociado a los electrones tiene el factor $\sqrt{m_e/m_i}$ el cual es pequeño.
Si $T_e \gg T_i$ el término de los iones también se vuelve pequeño pues el término exponencial tiende a cero a medida que su argumeto se vuelve más grande.\\
De lo anterior se conlcuye que un amortiguamiento fuerte de Landau ocurre cuando $T_i$ se aproxima a $T_e$ y por lo tanto las ondas acusticas de los iones solo pueden propagarse sin sufrir atenuaciones significativas cuando el plasma tiene la propiedad $T_e \gg T_i$
\section{Relaciones de potencia}
Siguiendo este estudio de ondas sería de especial interes entonces calcular la energía asociada al amortiguamiento de estas.
Pero como ya se vió en el capítulo \ref{Ch2:inestabilidades} existe un intercambio de energía entre las ondas  y las partículas por lo que en vez de calcular la energía sería más apropiado calcular la potencia asociada al amortiguamiento.\\
Si se asume que toda la energía de la onda se encuentra en el campo eléctrico se tiene entonces que la potencia que pierde la onda es:
\begin{equation}
\label{eq:potencia_perdida-onda_Landau}
\begin{split}
P_{\substack{perdida\\onda}}&\sim\frac{d}{dt}\left\langle \frac{\epsilon_0 E^2_{onda}}{2}\right\rangle \sim \frac{d}{dt}\left[\frac{\epsilon_0|E^2_{onda}|}{4}exp(-2|\omega_i|t)\right]\\
&=-\frac{|\omega_i|\epsilon_0 E^2_{onda}}{2}\\
&=\sqrt{\frac{\pi}{8}}\frac{\omega_{pe}}{2k^3\lambda^3_{De}}exp\left(-\omega^2/k^2v^2_{Te}\right)\epsilon_0 E^2_{onda}
\end{split}
\end{equation}
Donde $\langle E^2_{onda}\rangle =|E_{onda}|^2\langle cos^2(kx-\omega t)\rangle =|E_{onda}|/2$. Sin embargo, de la ecuación \ref{eq:promedio_w_total_f_maxweliana} se tenía que la potencia ganada por las partículas resonantes no atrapadas en una onda era:
\begin{equation}
\label{eq:P_ganada_particulas_Landau}
\begin{split}
P_{\substack{ganacia\\particulas} } &=-\frac{\pi m \omega }{2k^2} \left( \frac{q E_{onda}}{m} \right) ^2 \left[ \frac{d}{dv_0}f(v_0) \right]_{v_0=\omega / k} \\
&=-\frac{\pi m \omega}{2k^2}\left( \frac{qE_{onda}}{m} \right)^2 \left[\frac{d}{dv_0} \left\lbrace \left(\frac{m}{2\pi k_BT} \right)^{1/2}n_0 exp\left( -\frac{mv^2}{2k_BT}\right) \right\rbrace \right]_{v_0= \omega /k}\\
&=\frac{\pi m \omega}{2k^2}\left(\frac{qE_{onda}}{m}\right)^2 \left( \frac{m}{2\pi k_BT} \right)^{1/2}\left( \frac{m \omega}{k_BTk}\right)n_0 exp \left(-\frac{\omega^2}{kv^2_{Te}}\right)
\end{split}
\end{equation}
Si se hace la aproximación $\omega \sim \omega_{pe}$
\begin{equation}
\label{eq:P_ganancia_part_aprox}
\begin{split}
P_{\substack{ganacia\\particulas} } &= \sqrt{\frac{\pi}{8}}\sqrt{\frac{m}{k^3_BT^3}}\frac{\omega ^2_{pe}}{k^3}q^2n_0E^2_{onda}exp\left( -\frac{\omega ^2_{pe}}{k^2v^2_{Te}}\right)\\
&=\sqrt{\frac{\pi}{8}}\sqrt{\frac{m}{k_BT}}\frac{\omega ^2_{pe}}{k^3}\frac{\epsilon _0}{\lambda ^2_{De}}E^2_{onda}exp\left( -\frac{\omega ^2_{pe}}{k^2v^2_{Te}}\right)\\
&=\sqrt{\frac{\pi}{8}}\frac{\omega _{pe}}{k^3\lambda ^3_{De}}exp\left( -\frac{\omega ^2_{pe}}{k^2v^2_{Te}}\right) \epsilon _0 E^2_{onda}
\end{split}
\end{equation}
Que resulta ser la ecuación \ref{eq:potencia_perdida-onda_Landau} salvo un factor de 2. Esta discrepancia resulta de la suposición inicial de que toda la energía de la onda se encontraba contenida en el campo eléctrico. Lo que la ecuación \ref{eq:P_ganancia_part_aprox} ilustra es que el campo eléctrico solo aporta la mitad de la energía total encontrada en la onda. Queda entonces por determinar donde está contenida la otra mitad de la energía de la onda. No obstante, de esto se puede concluir que la tasa verdadera de perdida de potencia es dos veces la ecuación \ref{eq:potencia_perdida-onda_Landau}.
\section{Fórmula de Plemelj}
El método de Landau expuesto en este capítulo ilustró que la manera más adecuada de analizar problemas que llevan a la obtención de integrales mal definidas es proponiendo el problema como uno de valor inicial en lugar de un estacionario.
El resultado esencial de este método fue encontrar una manera de lidiar con las singularidades tomando el medio residuo del polo para determinar su contribución y el valor principal de la integral para el resto del contorno para el cual el polo no contribuye.
A pesar de ser una herramienta adecuada para lidiar con las singularidades, el método de Landau tiene la inconveniencia de ser algo tardado. Afortunadamente existe una manera de resumir su resultado esencial y se hace por medio de la llamada fórmula de Plemelj.
\begin{equation}
\label{eq:formula_Plemelj}
lim _{\epsilon \to 0}\frac{1}{\xi -a\mp i|\xi |}=P\frac{1}{\xi -a}\pm i\pi \delta (\xi -a)
\end{equation}
La cual muestra de manera simple como lidiar con integrales que contengan una singularidad de la forma en la que aparecen en la función de dispersión del plasma.
Esto ayuda a simplificar el proceso ya que en lugar de realizar el método de Landau completo se puede empezar por un análisis de Fourier y después utilizar la ecuación \ref{eq:formula_Plemelj} para poder lidiar con las singularidades que se puedan presentar y de esta manera evaluar las integrales.
Para mayor detalle sobre su derivación vease el apéndice \ref{Ap2:Plemelj}.
\section{Criterio de Penrose}
El análisis que se ha estando realizando hasta ahora ha mostrado que las ondas electrostáticas están sujetas al amortiguamiento de Landau,
el cual es una atenuación no colisional proporcional a $[\partial f / \partial u]_{u=\omega / k}$ y que a su vez es consistente con el cálculo de potencia recibida por las partículas desde una onda electrostática.\\
Debido a que una distribución Maxwelliana tiene pendiente negativa, su amortiguamiento de Landau asociado resulta ser siempre una atenuación de la onda.
Lo cual es consistente con lo desarrollado en el capítulo \ref{Ch2:inestabilidades} en donde se mostraba que la energía se transfería desde una onda hacia la partícula si habían más partículas lentas que rápidas con respecto a la velocidad de fase de la onda.
Queda por ver que sucede cuando la función de distribución no es Maxwelliana y en un particular una donde haya más partículas rápidas que lentas con respecto a la velocidad de fase.
En otras palabras, ver que pasa cuando $[\partial f / \partial u]_{u=\omega / k} > 0$.\\
Debido a que $f(u) \to 0$ a medida que $|u| \to \infty $, $f$ solo puede tener una pendiente postiva en un rango finito de velocidad postiva y de igual manera solo puede tener pendiente negativa en un rango finito de velocidad negativa.
En particular, se tiene que las pendientes postivas de la función de distribución deben estar siempre ubicadas a la izquierda de un máximo local de $f(u)$ en la región $u>0$ del plano velocidad-espacio.
Un máximo local en $f(u)$ corresponde a un haz de partículas veloces superpuestas sobre un fondo de partículas que tiene una $f(u)$ que decrece de manera monotona.
Queda entonces por ver si el proceso que da origen al amortiguamiento de Landau puede ser revertido en el sentido que genere una inestabilidad en las ondas.
Resulta que no solo eso sucede sino que tambiñen existe un criterio el cual muestra que tan fuerte debe de ser un haz para que de origen a una inestabilidad de Landau. A ese criterio se le conoce como criterio de Penrose. \cite{bellan2008fundamentals}\\
Se empieza entonces por recordar la ecuación \ref{eq:fourier_dispersion_sin_normalizar} la cual se reescribe de la siguiente manera
\begin{equation}
\label{eq:k2=Q}
k^2 =Q(z)
\end{equation}
donde
\begin{equation}
\label{eq:Q(z)_integral}
Q(z) = \frac{q^2}{m \epsilon _0}\int^{\infty}_{-\infty} \frac{1}{(u -z)}\frac{\partial f_0}{\partial u}du
\end{equation}
es una función compleja de la variable $z=\omega /k$, también compleja.El número de onda $k$ se asume como una cantidad real positiva y se utilizará la fórmula de Plemelj para lidiar con la singularidad del integrando.
Por suposición, se tiene que el lado izquierdo de \ref{eq:k2=Q} es siempre real y positivo para cualquier $k$. Por lo las soluciones que se buscan son cuando $Q(z)$es real y positiva.\\
$Q(z)$  puede ser interpretada como un mapeo de $z$-plano $\mapsto$ $Q$-plano. Debido a que las soluciones de \ref{eq:k2=Q} que producen inestabilidades son aquellas en donde $\operatorname{Im}(\omega)$, la parte superior del $z$-plano corresponde a las inestabilidades mientras que el eje rea de $z$ representa la división entre estabilidad e inestabilidad.\\
Considerese entonces un controno en línea recta $C_z$ paralelo al eje real de $z$ y que pasa ligeramente por encima de este .
Dicho contorno puede ser representado como $z=z_r + i\delta$ donde $d$ es una constante pequeña y $z_r$ toma valores de $-\infty$ a $\infty$.\\
La función $Q(z) \to 0$ cuando $z \pm \infty$ y por lo tanto a medida que $z$ se mueve a lo largo de $C_z$  el contorno $C_Q$ correspondiente en el $Q-plano$ complejo debe empezar y terminar su recorrido en el origen.
Además, como $Q$ puede ser evaluada por medio de la fórmula de Plemelj se observa que $Q$ es finita para todas las $z$ en $C_z$.
Se tiene entonces en consequencia que $C_Q$ es una curva continua finita que empieza y termina en el origen del $Q$-plano complejo.\\
Debido a que el mapeo de la parte superior del $z$-plano corresponde a la región encerrada por $C_Q$ se tiene que si $Q(z)$ no toma valor real positivos alguno de $z$ entonces  la $C_Q$ correspondiente no puede ser solución de \ref{eq:k2=Q} correspondiente a una inestabilidad. 
Se impone entonces la condición de que $Q(z)$ tome valores positivos de la parte real de $z$ lo que es equivalente a hacer que $C_Q$ pase por la región $\operatorname{Re}(Q) >0$ y cruce el eje real de $Q$.
Lo cual corresponde a soluciones que generen inestabilidades.
Por último, se recuerda que $C_z$ son la frecuencias ligeramente inestables por lo el mapeo a $C_Q$ donde se cruza el eje real positivo de $Q$ también representa inestabilidades ligeras.\\
Se presta atención entonces a cuando $C_Q$ cruza el eje positivo real de $Q$. Si se usa la fórmula de Plemelj en \ref{eq:Q(z)_integral} se obtiene
\begin{equation}
\operatorname{Im}(Q) = \frac{q^2}{m\epsilon _0}\pi \left[\frac{\partial f_0}{\partial u}\right]_{u= \omega /k}
\end{equation}
Si además se desplaza a lo largo de $C_Q$ desde un punto ligeramente por debajo del eje real de $Q$ a uno ligeramente por encima, $\operatorname{Im}(Q)$ pasa de ser negativo a positivo. Y en consecuencia $[\partial f_0 / \partial u]_{u = \omega /k}$ cambia de negativo a positivo de tal manera que sobre el eje real positivo de $Q$ $f_0$ es un mínimo en algún valor $u=u_{min}$ donde el subíndice $min$ indica que ese es el valor que resulta en un mínimo no que $u$ sea un mínimo.\\
Una expansión de Taylor alrededor de este mínimo da
\begin{equation}
\label{eq:taylor_min_Penrose}
f(u) =f[u_{min}+(u-u_{min})] = f(u_{min})+0+\frac{(u -u_{min})^2}{2}f''(u_{min})+ \cdots
\end{equation}
Y ya que $f(u_{min})$ es una constante, se puede escribir la derivada parcial de $f_0$ como
\begin{equation}
\frac{\partial f_0}{\partial u} = \frac{\partial}{\partial u}\left[f_0(u)-f_0(u_{min})\right]
\end{equation}
Lo cual facilita la integración por partes de \ref{eq:Q(z)_integral}
\begin{equation}
\label{eq:Q_z=u_min}
\begin{split}
Q\left( z=u_{min}\right) &=\frac{q^2}{m\epsilon _0}\left\lbrace P\int ^{\infty}_{-\infty} \frac{\frac{\partial}{\partial u}\left[f(u)-f(u_{min})\right]}{(u-u_{min})}du + i\pi \left[\frac{\partial}{\partial u}f(u)\right]_{u=u_{min}}\right\rbrace\\
&=\frac{q^2}{m\epsilon _0}P\int ^{\infty}_{-\infty} \frac{1}{(u-u_{min})^2}\left[ f(u)-f(u_{min})\right]du\\
&= \frac{q^2}{m\epsilon _0}\int ^{\infty}_{-\infty}\frac{1}{(u-u_{min})^2} \left[0+\frac{(u-u_{min})^2}{2}f''(u_{min})+\cdots \right]du
\end{split}
\end{equation}
En el segundo renglón de la ecuación \ref{eq:Q_z=u_min} se tiene que la parte imaginaria se hace cero  mientras que en el tercer renglón se ha quitado la $P$ debido a que el término dominante es proporcional a $(u-u_{min})^2$ y por lo tanto ya no hay singularidad en $u=u_{min}$.\\
Se tiene entonces que el requisito para que haya una ligera inestabilidad se puede resumir en que $f(u)$ tiene un mínimo en  $u-u_{min}$ y el valor de $Q$ es real y positivo, esto es:
\begin{equation}
\label{eq:criterio_Penrose}
Q(u_{min})=\frac{q^2}{m\epsilon _0}\int ^{\infty}_{-\infty}\left[\frac{f(u)-f(u_{min})}{(u-u_{min})^2}\right]du > 0
\end{equation}
Por último se tiene que el criterio de Penrose extiende el análisis de inestabilidad en dos corrientes a una distribución arbitraria que contiene un número finito de haces con distintas temperaturas.
%%%%%%%Anderson%%%%%%%%%%%%%%%
%Para evaluar la integral se hace uso de la teoría de variable compleja, en particular el teorema del residuo y el valor principal de Cauchy.\\
%Se reescribe la integral en una forma más conveniente.
%\begin{equation}
%\label{eq:disp_general_alt}
%1=\frac{\omega^2_{p0}}{k^2}\int^{\infty}_{-\infty}\frac{df/dv}{v- \omega /k}dv
%\end{equation}
%
%La manera de resolver el problema de la singularidad es tratar a la integral en la ecuación \ref{eq:disp_general_alt} como un problema de integral de contorno, donde el cotorno con el que se trabajará será sobre el eje real de $v$, pues se está trabajando en el plano complejo de $v$, y un semicírculo que pase por debajo del polo $\omega /k$ \cite{hassani2013mathematical}. Se tiene entonces que:
%\begin{equation}
%\int_{c}\frac{df/dv}{v- \omega /k}dv= \int^{\omega/k -\epsilon}_{-\infty}\frac{df/dv}{v- \omega /k}dv + \int^{\infty}_{\omega/k +\epsilon}\frac{df/dv}{v- \omega /k}dv + \int_{c_0}\frac{df/dv}{v- \omega /k}dv
%\end{equation}
%Al tomar el límite cuando $\epsilon \rightarrow 0$ los dos primeros términos del lado derecho resultan ser el valor princial de la integral en la ecuación \ref{eq:disp_general_alt}. Por otro lado, se tiene que en en el contorno $c_0$ la magnitud de $v-\omega/k$ es constante por lo que se puede expresar como $v-\omega/k =\epsilon e^{i\theta}$ y $dv= i\epsilon e^{i\theta}d\theta$. De manera que se tiene \cite{hassani2013mathematical}:
%\begin{equation}
%\int_{c_0}\frac{df/dv}{v- \omega /k}dv = i \pi \left(\frac{df}{dv}\right)_{v=\omega/k}
%\end{equation}
%Con esto, se reescribe la ecuación \ref{eq:disp_general_alt} como:
%\begin{equation}
%\label{eq:disp_general_residuo}
%1 = \frac{\omega^2_{p0}}{k^2} \left(PV\int^{\infty}_{-\infty}\frac{df/dv}{v- \omega /k}dv + i \pi \left(\frac{df}{dv}\right)_{v=\omega/k}\right)
%\end{equation}
%Asumiendo el caso donde la velocidad de fase de la onda es mucho mayor que las velocidades en la distribución el valor principal ($PV$) se puede aproximar expandiendo el denominador de la integral alrededor de $v=0$, esto da:
%\begin{equation}
%PV= \frac{1}{\omega}\int^{\infty}_{-\infty} \left(1+\frac{kv}{\omega}+\frac{k^2v^2}{\omega^2}+...\right)\frac{df}{dv}dv
%\end{equation}
%Y haciendo uso de las propiedades de $f(v)$ como función de distribución \cite{anderson2001tutorial}
%\begin{equation}
%\int^{\infty}_{-\infty}f(v)dv=1
%\end{equation}
%\begin{equation}
%\int^{\infty}_{-\infty}vf(v)dv=0
%\end{equation}
%\begin{equation}
%\int^{\infty}_{-\infty}v^2f(v)dv=\frac{1}{2}v^2_T
%\end{equation}
%Se tiene entonces que el valor princial es:
%\begin{equation}
%PV\int^{\infty}_{-\infty}\frac{df/dv}{v- \omega /k}dv \approx -\frac{k}{\omega^2}\left(1+\frac{3}{2}\frac{k^2v^2_T}{\omega^2}\right)
%\end{equation}
%Por lo que la relación de dispersión se puede expresar como:
%\begin{equation}
%\label{eq:disp_var_compleja}
%1=\frac{\omega^2_{p0}}{\omega^2}\left(1+\frac{3}{2}\frac{k^2v^2_T}{\omega^2}\right) + i\pi \frac{\omega^2_{p0}}{k^2}\left(\frac{df}{dv}\right)_{v=\omega/k}
%\end{equation}
%Adicionalmente, si se aproxima al caso $\omega^2 \approx \omega^2_{p0}$ la ecuación \ref{eq:disp_var_compleja} se cambia a:
%\begin{equation}
%\omega^2=\omega^2_{p0}+\frac{3}{2}k^2v^2_T+i\pi\frac{\omega^4_{p0}}{k^2}\left(\frac{df}{dv}\right)_{v=\omega/k}
%\end{equation}
%Es de interes entonces la parte imaginaria de $\omega$. Para esto, se recuerda que $\omega/k$ es lo suficientemente grande en comparación con las velocidades de la distribución que la derivada de la función de distribución es lo suficientemente pequeña para realizar una aporximación por series. Adicionalmente, se desprecia la corrección térmica de la parte real \cite{chenintroplasmas}, de manera que se tiene
%\begin{equation}
%\omega^2 \approx \omega^2_{p0}\left(1+i\pi\frac{\omega^2_{p0}}{k^2}\left(\frac{df}{dv}\right)_{v=\omega/k}\right)
%\end{equation}
%Al realizar una aproximación en series para encontar $\omega_{im}$ se llega a:
%\begin{equation}
%\omega_{im}\approx \frac{\pi \omega^2_{p0}}{2k^2}\left(\frac{df}{dv}\right)_{v=\omega/k}
%\end{equation}
%Ahora bien si $f(v)$ es una distribución maxwelliana entonces $f(v) \propto exp(-v^2/v^2_T)$, es decir una función decreciente en la velocidad por lo que su derivada evaluada en $\omega/k$ es negativa y entonces $\omega_{im}$ representa un amortiguamiento en la onda.
%A este amortiguamineto se le conoce como amortiguamiento de Landau, cuyo origen resulta evidente si se recuerda que la suposición de una velocidad de fase lo suficientemente grande implica que las especies se mueven a una velocidad más lenta y por lo tanto la transferencia de energía es desde la onda hacia las partículas.
\onlyinsubfile{\bibliographystyle{unsrt}}
\onlyinsubfile{\bibliography{../referencias}}
\end{document}