\documentclass[../tesis_main_file.tex]{subfiles}
\begin{document}
\onlyinsubfile{\appendix}
\onlyinsubfile{\pagenumbering{arabic}}
\onlyinsubfile{\chapter{Cálculo de raices}}
%\section{Cálculo de raices}\label{Ap:raices}
\subsection*{Resolviendo ecuaciones cuárticas.}
Un método para resolver ecuaciones cuárticas es el llamado método de Ferrari \cite{higheralgebra} y prosigue de la siguiente manera. Si se parte de la forma general de la ecuación cuártica, es decir:
\begin{equation}
z^4 +az^3+bz^2+cz+d=0
\label{eq:general_cuartica}
\end{equation}
Entonces, se propone el cambio de variable $z=x-\frac{a}{4}$, lo que da como resultado:
\begin{multline}
x^4-ax^3+\frac{3}{8}a^2x^2-\frac{1}{16}a^3x+\frac{1}{256}a^4 +ax^3 - \frac{3}{4}a^2x^2 + \frac{3}{16}a^3x \\
-\frac{1}{64}a^4 + bx^2 -\frac{1}{2}abx+ \frac{1}{16}a^2b + cx -\frac{1}{4}ac +d =0
\end{multline}
Que al agrupar términos queda:
\begin{equation}
x^4+\left(b-\frac{3}{8}a^2\right)x^2 + \left(\frac{1}{8}a^3-\frac{1}{2}ab +c\right)x+ \left(d-\frac{1}{4}ac +\frac{1}{16}a^2b-\frac{3}{256}a^4\right)
\end{equation}
Se llega entonces a una ecuación cuártica reducida, de la forma:
\begin{equation}
x^4 +px^2+qx+r=0
\label{eq:cuartica_reducida}
\end{equation}
El siguiente paso en introducir un término auxiliar $\alpha$, por lo que se reescribe la ecuación \ref{eq:cuartica_reducida} como:
\begin{equation}
x^4 +px^2+qx+r=\left(x^2 +\frac{p}{2}+\alpha\right)^2 +qx+r -\frac{p^2}{4}-\alpha^2 -2x^2\alpha -p\alpha=0
\end{equation}
O bien:
\begin{equation}
\label{eq:cuartica_alpha}
\left(x^2 +\frac{p}{2}+\alpha\right)^2 - \left[2\alpha x^2 -qx + \left(\alpha^2 +p\alpha-r +\frac{p^2}{4}\right)\right]=0
\end{equation}
Entonces, se escoge el valor de $\alpha$ tal que complete el cuadrado dentro de los corchetes. Esto es que tenga una raíz doble, por lo que su discriminante sería cero. Se requiere entonces que $\alpha$ cumpla con:
\begin{equation}
\label{eq:discriminante_cuartica_reducida}
q^2 -4 \cdot 2\alpha \left(\alpha^2 +p\alpha-r +\frac{p^2}{4}\right)=0
\end{equation}
La cual es una ecuación cúbica con tres raíces. Tomamos entonces una de esas raíces, por ejemplo $\alpha_0$. Se tiene entonces que la raíz dentro de los corchetes es $q/4\alpha_0$. Por lo que la ecuación \ref{eq:cuartica_alpha} se reescribe como:
\begin{equation}
\left(x^2 +\frac{p}{2}+\alpha\right)^2 -2\alpha_0 \left(x -\frac{q}{4\alpha_0}\right)^2=0
\end{equation}
La cual es una diferencia de cuadrados. Entonces podemos ver ecuación anterior como:
\begin{equation}
\left(x^2 + \frac{p}{2}+\alpha_0 - \sqrt{2}\alpha_0 \left(x-\frac{q}{4\alpha_0}\right)\right)\left(x^2 + \frac{p}{2}+\alpha_0 + \sqrt{2}\alpha_0 \left(x-\frac{q}{4\alpha_0}\right)\right)=0
\end{equation}
Se llega entonces a dos raíces cuadráticas:
\begin{equation}
\label{eq:cuartica_cuadratica1}
x^2 -\sqrt{2\alpha_0}x+\left(\frac{p}{2}+\alpha_0 + \frac{q}{2\sqrt{2\alpha_0}}\right)=0
\end{equation}
\begin{equation}
\label{eq:cuartica_cuadratica2}
x^2 +\sqrt{2\alpha_0}x+\left(\frac{p}{2}+\alpha_0 - \frac{q}{2\sqrt{2\alpha_0}}\right)=0
\end{equation}
De las cuales se pueden encontrar sus dos raíces, que debido a que se llegó a \ref{eq:cuartica_cuadratica1} y a \ref{eq:cuartica_cuadratica2} por medio de identidades, resultan ser raíces de la ecuación \ref{eq:cuartica_reducida}, de la cual se pueden recuperar las raíces de \ref{eq:general_cuartica} al aplicar el cambio de variable.\\
La forma explícita de las raíces de \ref{eq:cuartica_cuadratica1} y \ref{eq:cuartica_cuadratica2} son:
\begin{equation}
x = \frac{1}{2}\left(\sqrt{2\alpha_0} \pm \sqrt{2\alpha_0 - 4\left(\frac{p}{2} + \alpha_0 + \frac{q}{2\sqrt{2\alpha_0}}\right)}\right)
\end{equation}
\begin{equation}
x = \frac{1}{2}\left(-\sqrt{2\alpha_0} \pm \sqrt{2\alpha_0 - 4\left(\frac{p}{2} + \alpha_0 - \frac{q}{2\sqrt{2\alpha_0}}\right)}\right)
\end{equation}
Que al aplicar el cambio de variable resulta en:
\begin{equation}
\label{eq:raiz_12_cuartica}
z_{1,2}= \frac{1}{2} \left(-\frac{a}{2} + \sqrt{2\alpha_0} \pm \sqrt{2\alpha_0 - 4\left(\frac{p}{2} + \alpha_0 + \frac{q}{2\sqrt{2\alpha_0}}\right)}\right)
\end{equation}
\begin{equation}
\label{eq;raiz_34_cuartica}
z_{3,4}= \frac{1}{2} \left(-\frac{a}{2} - \sqrt{2\alpha_0} \pm \sqrt{2\alpha_0 - 4\left(\frac{p}{2} + \alpha_0 - \frac{q}{2\sqrt{2\alpha_0}}\right)}\right)
\end{equation}
Donde se recuerda que 
\begin{align}
\label{eq:valores_p_cuartica}
p&=\left(b-\frac{3}{8}a^2\right)\\
\label{eq:valores_q_cuartica}
q&=\left(\frac{1}{8}a^3 - \frac{1}{2}ab + c \right)
\end{align}
La expresión explícita de $\alpha_0$ se obtiene a partir de la llamada ecuación de Cardan \cite{higheralgebra}. El procedimiento es el siguiente:\\
Partiendo de la ecuación \ref{eq:discriminante_cuartica_reducida}, la cual es una ecuación cúbica.
\begin{equation}
\alpha_0^3 + p\alpha_0^2+ \alpha_0 \left(\frac{p^2}{4}-r\right) - \frac{q^2}{8}=0
\end{equation}
Se vuelve a usar un cambio de variable similar al de la ecuación cuártica de la forma $\alpha_0 = y- \frac{p}{3}$.
\begin{equation}
\label{eq:cubica_camb_var}
y^3 +y\left(-\frac{p^2}{12}-r\right) + \left(\frac{pr}{3}-\frac{q^2}{8}-\frac{p^3}{108}\right)=0
\end{equation}
Ahora bien, si nombramos
\begin{align}
\label{eq:pp_alpha}
p'&=\left(-\frac{p^2}{12}-r\right)\\
\label{eq:qp_alpha}
q'&=\left(\frac{pr}{3}-\frac{q^2}{8}-\frac{p^3}{108}\right)
\end{align}
La ecuación \ref{eq:cubica_camb_var} se puede reescribir entonces como:
\begin{equation}
\label{eq:cub_red}
y^3 +p'y+q'=0
\end{equation}
Donde la ecuación de Cardan \cite{higheralgebra} índica que las raíces son de la forma:
\begin{equation}
y = \left(-\frac{q'}{2} + \sqrt{\frac{q^{'2}}{4}+\frac{p^{'3}}{27}}\right)^{1/3} + \left(-\frac{q'}{2} - \sqrt{\frac{q^{'2}}{4}+\frac{p^{'3}}{27}}\right)^{1/3}
\end{equation}
Entonces, al recuperar el cambio de variable se tiene:
\begin{equation}
\label{eq:alpha_cero}
\alpha_0 = \left(-\frac{q'}{2} + \sqrt{\frac{q^{'2}}{4}+\frac{p^{'3}}{27}}\right)^{1/3} + \left(-\frac{q'}{2} - \sqrt{\frac{q^{'2}}{4}+\frac{p^{'3}}{27}}\right)^{1/3} - \frac{p}{3}
\end{equation}
Se recuerda también que:
\begin{equation}
r= \left(d-\frac{1}{4}ac +\frac{1}{16}a^2b-\frac{3}{256}a^4\right)
\end{equation}
Por último, cabe mencionar que $\alpha_0$ se obtiene de la suma de dos raíces cúbicas, es decir de la forma:
\begin{equation}
\alpha_0 = s +t - \frac{p}{3}
\end{equation}
Se tiene entonces que para obtener un valor válido de $\alpha_0$ solo se pueden escoger ciertos valores de los tres posibles para $s$ y $t$. Estos valores son:
\begin{align}
\label{eq:raices_permitidas_para_aplha}
\alpha_0&=s_1+t_1- \frac{p}{3}\\
\alpha_0&=s_2+t_3- \frac{p}{3}\\
\alpha_0&=s_3+t_2- \frac{p}{3}
\end{align}
\subsection*{Caso misma $\omega_{p\sigma}$ con una especie incidente y la otra estacionaria.}
Expandiendo la ecuación \ref{eq:misma_omega_reposo} se obtiene:
\begin{equation}
\label{eq:misma_omega_expand}
z^4 -2\lambda z^3 +(\lambda^2 -2)z^2 + 2\lambda z -\lambda^2=0
\end{equation}
Para resolver está ecuación se sigue entonces el método de Ferrari, donde se empieza por el cambio de variable $z=x+\frac{\lambda}{2}$, para obtener una expresión simplificada
\begin{equation}
x^4 - \frac{1}{2}(4 + \lambda^2)x^2 + \frac{1}{16}\lambda^2 (\lambda^2 -8)=0
\end{equation}
Donde, al recordar la ecuación \ref{eq:valores_q_cuartica} se tiene que el término de primer orden se cancela. Pues $q=-\lambda^3 + \lambda (\lambda^2 -2) +\lambda$. De esto, se obtiene lo que resulta ser una ecuación bicuadrática, es decir una expresión cuadrática para $x^2$. Cuyas raíces son entonces:
\begin{equation}
x^2 = \frac{1}{4}(4 + \lambda^2 \pm 4\sqrt{ \lambda^2 +1})
\end{equation}
Por lo que las raíces cuárticas son:
\begin{equation}
x= \pm \frac{1}{2}\sqrt{4 + \lambda^2 \pm 4\sqrt{ \lambda^2 +1}}
\end{equation}
Que recordando el cambio de variable se recupera entonces las raíces para la ecuación \ref{eq:misma_omega_expand}.
\begin{equation}
z = \frac{1}{2}\left(\lambda \pm \sqrt{4 + \lambda^2 \pm 4\sqrt{ \lambda^2 +1}}\right)
\end{equation}
Ahora bien, la elección de signos es de acuerdo a la necesidad de una solución con su parte imaginaria positiva, la cual se obtiene tomando el primer signo como positivo y el segundo como negativo.
\subsection*{Caso plasma estacionario y plasma incidente (cuatro especies)}
Al expandir la ecuación \ref{eq:disp_d-d} se obtiene:
\begin{equation}
z^4 -2\lambda z^3+ (\lambda^2 -2 - 2\epsilon_1)z^2 + (2\lambda+\epsilon_1 \lambda)z+ \lambda^2 (-1 -\epsilon_1)=0
\end{equation}
Y nuevamente haciendo la sustitución $z=x + \frac{ \lambda}{2}$ se obtiene:
\begin{multline}
x^4+2x^3\lambda + \frac{3}{2}x^2 \lambda^2 + \frac{1}{2}x\lambda^3 +\frac{1}{16}\lambda^4 -2x^3 \lambda -3x^2 \lambda^2 - \frac{3}{2}x\lambda^3 - \frac{1}{4}\lambda^4 +x^2\lambda^2 + x\lambda^3 +\frac{1}{4}\lambda^4\\ 
  -2x^2-2x\lambda -\frac{1}{2}\lambda^2 - 2x^2 \epsilon_1-2x\lambda\epsilon_1-\frac{1}{2}\lambda^2\epsilon_1+2x\lambda+2x\epsilon_1\lambda + \lambda^2 \epsilon_1\lambda^2+\lambda^2(-1-\epsilon_1)=0
\end{multline}
La cual se puede reescribir como:
\begin{equation}
x^4 -\frac{1}{2}(\lambda^2 +4\epsilon_1 + 4)x^2 + \frac{1}{16}\lambda^2(\lambda^2 - 8\epsilon -8)=0
\end{equation}
La cual resulta una ecuación bicuártica, se puede entonces sacar para $x^2$ el siguiente resultado:
\begin{equation}
x^2 = \frac{1}{4}\left(4+4\epsilon_1+\lambda^2 \pm 4\sqrt{1+2\epsilon_1+\epsilon_1^2 + \lambda^2+\epsilon_1\lambda^2}\right)
\end{equation}
Se tiene entonces que $x$ es:
\begin{equation}
x=\pm \frac{1}{2} \sqrt{4+4\epsilon_1+\lambda^2 \pm 4\sqrt{1+2\epsilon_1+\epsilon_1^2 + \lambda^2+\epsilon_1\lambda^2}}
\end{equation}
Y recuperando la variable $z$ se tiene que las raíces son de la forma:
\begin{equation}
z = \frac{1}{2}\left(\lambda \pm \sqrt{4+4\epsilon_1+\lambda^2 \pm 4\sqrt{1+2\epsilon_1+\epsilon_1^2 + \lambda^2+\epsilon_1\lambda^2}}\right)
\end{equation}
\subsection*{Caso con especies de $\omega_{p\sigma}$ diferentes con una incidente y la otra en reposo.}
La ecuación a resolver es de la forma:
\begin{equation}
\label{eq:dif_omega_reposo_incidente_cuartica}
z^4 - 2\lambda z^3 + z^2(\lambda^2 -\epsilon -1) +2\epsilon \lambda z - \lambda^2
\epsilon =0
\end{equation}
Entonces, al aplicar el cambio de variable $z=x+\frac{\lambda}{2}$ queda:
\begin{equation}
x^4 + (-\frac{\lambda^2}{2} -\epsilon -1)x^2 + (\epsilon \lambda - \lambda)x + \frac{\lambda^4}{16}-\frac{\epsilon \lambda^2}{4}-\frac{\lambda^2}{4}=0
\end{equation}
Donde los coeficientes $p$,$q$ y $r$ son entonces:
\begin{align}
p&=-\frac{\lambda^2}{2} -\epsilon -1\\
q&=\epsilon \lambda - \lambda\\
r&=\frac{\lambda^4}{16}-\frac{\epsilon \lambda^2}{4}-\frac{\lambda^2}{4}
\end{align}
A diferencia de los casos anteriores, la ecuación \ref{eq:dif_omega_reposo_incidente_cuartica} no se reduce a una bicuadrática por lo que se tiene que las raíces de $z$ están expresadas ya sea por la ecuación \ref{eq:raiz_12_cuartica} o por la ecuación \ref{eq;raiz_34_cuartica} que se encuentran expresadas en términos de los coeficientes $p$,$q$ y un término $\alpha_0$.\\
El siguiente paso es entonces encontrar una expresión explícita para $\alpha_0$ la cual debe ser de la misma forma que la ecuación \ref{eq:alpha_cero} que a su vez está expresada en términos de los coeficientes $p'$ y $q'$ los cuales están definidos por las ecuaciones \ref{eq:pp_alpha} y \ref{eq:qp_alpha}. Para este caso se tiene entonces que $p'$ y $q'$ son:
\begin{align}
\label{eq:pp_alpha_diferente_omega_reposo}
p'&= -\frac{1}{12}\left(1+\epsilon-\lambda^2\right)^2\\
\label{eq:qp_alpha_diferente_omega_reposo}
q'&= \frac{1}{108} \left[ \epsilon^3 -3\epsilon^2 (\lambda^2
-1)+3\epsilon(\lambda^4 + 16\lambda^2 +1)- (\lambda^2 -1)^3 \right]
\end{align}
$\alpha_0$ es entonces:
\begin{multline}
\label{eq:alpha_diferentes_omegas_reposo}
\alpha_0 = \frac{1}{3}\left(\frac{\lambda^2}{2} + \epsilon + 1 \right)\\
+ \left(\frac{1}{216}\lbrace-\epsilon^3+3\epsilon^2(\lambda^2-1)-
3\epsilon(\lambda^4 +16\lambda^2 +1)+ (\lambda^2 -1)^3\rbrace \right. \\
\left. +\frac{\lambda}{12\sqrt{3}}\lbrace\epsilon [\epsilon^3-3\epsilon^2(\lambda^2 -1)+3\epsilon(\lambda^4 + 7\lambda^2 +1)- (\lambda^2-1)^3]\rbrace^{1/2} \vphantom{\frac{1}{216}}\right)^{1/3}\\
+ \left(\frac{1}{216}\lbrace-\epsilon^3+3\epsilon^2(\lambda^2-1)-
3\epsilon(\lambda^4 +16\lambda^2 +1)+ (\lambda^2 -1)^3\rbrace \right. \\
\left. -\frac{\lambda}{12\sqrt{3}}\lbrace\epsilon [\epsilon^3-3\epsilon^2(\lambda^2 -1)+3\epsilon(\lambda^4 + 7\lambda^2 +1)- (\lambda^2-1)^3]\rbrace^{1/2} \vphantom{\frac{1}{216}}\right)^{1/3}
\end{multline}
En caso de que los términos elevados a la $1/3$ sean complejos, nótese que son el conjugado del otro, de esta manera, la expresión \ref{eq:raices_permitidas_para_aplha} nos permite seleccionar las raíces de estos términos de tal manera que se cancele su parte imaginaria haciendo así a $\alpha_0$ un número real.\\
Recordando entonces que la raíz de interes es aquella cuya parte imaginaria sea positiva, se escogen los signos postivos en la ecuación \ref{eq:raiz_12_cuartica} y se sutituyen los coeficientes $p$ y $q$. Queda entonces que la raíz es:
\begin{equation}
\label{eq:raiz_diferente_omega_reposo}
z= \frac{1}{2} \left( \lambda + \sqrt{2\alpha_0} + \sqrt{2\alpha_0 -4 \left(-\frac{1}{2}(\frac{\lambda^2}{2}+\epsilon+1)+\alpha_0 +\frac{\epsilon \lambda - \lambda}{2\sqrt{2\alpha_0}}\right)}\right)
\end{equation}
Donde $\alpha_0$ no se sustituye por simplicidad.
\onlyinsubfile{\bibliographystyle{unsrt}}
\onlyinsubfile{\bibliography{../referencias}}
\end{document}